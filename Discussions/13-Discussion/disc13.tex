\documentclass{article}

% packages for math
\usepackage{amsthm}
\usepackage{amsmath}
\usepackage{amssymb}
\usepackage{amsfonts}

\usepackage{enumitem}

% package for including images
\usepackage{graphicx}

% TAKEN FROM OVERLEAF DOCUMENTATION
% https://www.overleaf.com/learn/latex/Code_listing
\usepackage{listings}
\lstset{language=Python}
\usepackage{xcolor}
\definecolor{codegreen}{rgb}{0,0.6,0}
\definecolor{codegray}{rgb}{0.5,0.5,0.5}
\definecolor{codepurple}{rgb}{0.58,0,0.82}
\definecolor{backcolour}{rgb}{0.95,0.95,0.92}
\lstdefinestyle{mystyle}{
  backgroundcolor=\color{backcolour},
  commentstyle=\color{codegreen},
  keywordstyle=\color{magenta},
  numberstyle=\tiny\color{codegray},
  stringstyle=\color{codepurple},
  basicstyle=\ttfamily\footnotesize,
  breakatwhitespace=false,
  breaklines=true,
  captionpos=b,
  keepspaces=true,
  numbers=left,
  numbersep=5pt,
  showspaces=false,
  showstringspaces=false,
  showtabs=false,
  tabsize=2
}
\lstset{style=mystyle}

\newcommand{\vv}[1]{\mathbf{#1}}
\newcommand{\R}{\mathbb R}
\DeclareMathOperator{\vspan}{span}
\DeclareMathOperator{\cod}{cod}
\DeclareMathOperator{\ran}{ran}
\DeclareMathOperator{\col}{Col}
\DeclareMathOperator{\nul}{Nul}
\DeclareMathOperator{\rank}{rank}

% environment for solutions
\theoremstyle{remark}
\newtheorem*{solution}{Solution}

% capital letters for problem parts
\renewcommand{\theenumi}{\Alph{enumi}}

% no page numbers
\pagenumbering{gobble}

% UNCOMMENT IF YOU DON'T WANT PROBLEMS ON INDIVIDUAL PAGES
% \renewcommand{\pagebreak}{}

\title{
  Week 13 Discussion
}
\author{CAS CS 132: Geometric Algorithms}
\date{November 27, 2023}

\begin{document}
\maketitle

\noindent During discussion sections, we will go over three problems.
\begin{itemize}
\item The first will be a warm-up question, to help you verify your understanding of the material.
\item The second will be a solution to a problem on the assignment of the previous week. \textbf{But not this week, there was no assignment last week.}
\item The third will be a problem similar to one on the assignment of the following week.
\end{itemize}
The remainder of the time will be dedicated to open Q\&A.

\pagebreak
\section{Inner Products, Norms, Orthogonality}

The first two problems come from \textit{Linear Algebra and its Applications}.

\begin{enumerate}
\item Given vectors
  \begin{displaymath}
    \vv u =
    \begin{bmatrix}
      2 \\ -5 \\ -1
    \end{bmatrix}
    \qquad
    \vv v =
    \begin{bmatrix}
      -7 \\ -4 \\ 6
    \end{bmatrix}
  \end{displaymath}
  compute (by hand) $\vv u \cdot \vv v$, $\|\vv u\|^2$, $\|\vv v\|^2$, and $\|\vv v + \vv u\|^2$
\item
  Show that if $\vv y$ is orthogonal to $\vv u$ and $\vv v$, then it is orthogonal to $\vv u + \vv v$.
\item Find a linearly independent set of three nonzero vectors in $\R^4$, all of which are orthogonal to
  \begin{displaymath}
    \begin{bmatrix}
      1 \\ 3 \\ -1 \\ 3
    \end{bmatrix}
  \end{displaymath}
\end{enumerate}

\medskip
\begin{solution}
\end{solution}

\pagebreak
\section{Damping without a Matrix}

Suppose that $A$ is a $n \times n$ stochastic matrix.
According to what we talked about in lecture, the matrix we use to perform PageRank is
\begin{displaymath}
  (1 - \alpha)A + \frac{\alpha \vv 1^{n \times n}}{n}
\end{displaymath}
where $\vv 1^{n \times n}$ here is the $n \times n$ all-ones matrix.
However, for the assignment this week, you will not be able to build this matrix because it is too dense.
Given a vector $\vv v$ from $\R^n$, write down an expression for
\begin{displaymath}
  \left((1 - \alpha)A + \frac{\alpha \vv 1^{n \times n}}{n}\right)\vv v
\end{displaymath}
which does not require building the matrix $\vv 1^{n \times n}$.
\textit{Hint.} The expression should be of the form
\begin{displaymath}
  (1 - \alpha)A\vv v + \vv u
\end{displaymath}
where $\vv u$ is a vector depending on $\vv v$ and the all-ones \textit{vector} $\vv 1^n \in \R^n$.

\begin{solution}
\end{solution}
\end{document}
