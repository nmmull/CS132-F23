\documentclass{article}

% packages for math
\usepackage{amsthm}
\usepackage{amsmath}
\usepackage{amssymb}
\usepackage{amsfonts}

\usepackage{enumitem}

% package for including images
\usepackage{graphicx}

% TAKEN FROM OVERLEAF DOCUMENTATION
% https://www.overleaf.com/learn/latex/Code_listing
\usepackage{listings}
\lstset{language=Python}
\usepackage{xcolor}
\definecolor{codegreen}{rgb}{0,0.6,0}
\definecolor{codegray}{rgb}{0.5,0.5,0.5}
\definecolor{codepurple}{rgb}{0.58,0,0.82}
\definecolor{backcolour}{rgb}{0.95,0.95,0.92}
\lstdefinestyle{mystyle}{
  backgroundcolor=\color{backcolour},
  commentstyle=\color{codegreen},
  keywordstyle=\color{magenta},
  numberstyle=\tiny\color{codegray},
  stringstyle=\color{codepurple},
  basicstyle=\ttfamily\footnotesize,
  breakatwhitespace=false,
  breaklines=true,
  captionpos=b,
  keepspaces=true,
  numbers=left,
  numbersep=5pt,
  showspaces=false,
  showstringspaces=false,
  showtabs=false,
  tabsize=2
}
\lstset{style=mystyle}

\newcommand{\vv}[1]{\mathbf{#1}}
\newcommand{\R}{\mathbb R}
\DeclareMathOperator{\vspan}{span}
\DeclareMathOperator{\cod}{cod}
\DeclareMathOperator{\ran}{ran}
\DeclareMathOperator{\col}{Col}
\DeclareMathOperator{\nul}{Nul}
\DeclareMathOperator{\rank}{rank}

% environment for solutions
\theoremstyle{remark}
\newtheorem*{solution}{Solution}

% capital letters for problem parts
\renewcommand{\theenumi}{\Alph{enumi}}

% no page numbers
\pagenumbering{gobble}

% UNCOMMENT IF YOU DON'T WANT PROBLEMS ON INDIVIDUAL PAGES
% \renewcommand{\pagebreak}{}

\title{
  Week 11 Discussion
}
\author{CAS CS 132: Geometric Algorithms}
\date{November 13, 2023}

\begin{document}
\maketitle

\noindent During discussion sections, we will go over three problems.
\begin{itemize}
\item The first will be a warm-up question, to help you verify your understanding of the material.
\item The second will be a solution to a problem on the assignment of the previous week.
\item The third will be a problem similar to one on the assignment of the following week.
\end{itemize}
The remainder of the time will be dedicated to open Q\&A.

\pagebreak
\section{Eigenvalues, Eigenvectors, Eigenspaces}

\begin{enumerate}
\item Find an invertible $2 \times 2$ matrix with no eigenvalues.
\item
  Let $T : \mathbb R^3 \to \mathbb R^3$ be the linear transformation which projects points onto the $x_1x_2$-plane.
  Find the eigenvalues and bases for the corresponding eigenspaces of the matrix implementing this transformation \textit{without doing any calculations}.
  Then write down the matrix implementing this transformation and find its characteristic polynomial.
  Check that the eigenvalues you get from the characteristic polynomial are the same.
\item
  Find the eigenvalues and bases for the corresponding eigenspace of
  \begin{displaymath}
    \begin{bmatrix}
      1 & -4 \\
      -3 & 5
    \end{bmatrix}
  \end{displaymath}

\end{enumerate}

\begin{solution}
\end{solution}

\pagebreak
\section{Complement of the Column Space}

Let $A$ be a $5 \times n$ matrix such that $\rank A = 4$, which has an LU decomposition where
\begin{displaymath}
  L =
  \begin{bmatrix}
    1 & 0 & 0 & 0 & 0 \\
    -1 & 1 & 0 & 0 & 0\\
    0 & 4 & 1 & 0 & 0\\
    2 & 0 & 0 & 1 & 0 \\
    0 & 3 & -3 & 0 & 1
  \end{bmatrix}
\end{displaymath}
Determine if $\vv v$ in $\col A$, where
\begin{displaymath}
  \vv v =
  \begin{bmatrix}
    2 \\ -5 \\ -11 \\ 5 \\ -12
  \end{bmatrix}
\end{displaymath}

\begin{solution}
\end{solution}

\pagebreak
\section{Characteristic Polynomials}

Find the characteristic polynomial for the matrix
\begin{displaymath}
  A =
  \begin{bmatrix}
    1 & -1 & 5 \\
    0 & 2 & 4 \\
    0 & 1 & 5
  \end{bmatrix}
\end{displaymath}
Use this to determine the eigenvalues of $A$.

\begin{solution}
\end{solution}

\end{document}
