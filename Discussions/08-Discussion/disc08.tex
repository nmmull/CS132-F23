\documentclass{article}

% packages for math
\usepackage{amsthm}
\usepackage{amsmath}
\usepackage{amssymb}
\usepackage{amsfonts}

\usepackage{enumitem}

% package for including images
\usepackage{graphicx}

% TAKEN FROM OVERLEAF DOCUMENTATION
% https://www.overleaf.com/learn/latex/Code_listing
\usepackage{listings}
\lstset{language=Python}
\usepackage{xcolor}
\definecolor{codegreen}{rgb}{0,0.6,0}
\definecolor{codegray}{rgb}{0.5,0.5,0.5}
\definecolor{codepurple}{rgb}{0.58,0,0.82}
\definecolor{backcolour}{rgb}{0.95,0.95,0.92}
\lstdefinestyle{mystyle}{
  backgroundcolor=\color{backcolour},
  commentstyle=\color{codegreen},
  keywordstyle=\color{magenta},
  numberstyle=\tiny\color{codegray},
  stringstyle=\color{codepurple},
  basicstyle=\ttfamily\footnotesize,
  breakatwhitespace=false,
  breaklines=true,
  captionpos=b,
  keepspaces=true,
  numbers=left,
  numbersep=5pt,
  showspaces=false,
  showstringspaces=false,
  showtabs=false,
  tabsize=2
}
\lstset{style=mystyle}

\newcommand{\vv}[1]{\mathbf{#1}}
\newcommand{\vspan}{\mathsf{span}}
\newcommand{\ran}{\mathsf{ran}}
\newcommand{\cod}{\mathsf{cod}}
\newcommand{\R}{\mathbb R}

% environment for solutions
\theoremstyle{remark}
\newtheorem*{solution}{Solution}

% capital letters for problem parts
\renewcommand{\theenumi}{\Alph{enumi}}

% no page numbers
\pagenumbering{gobble}

% UNCOMMENT IF YOU DON'T WANT PROBLEMS ON INDIVIDUAL PAGES
% \renewcommand{\pagebreak}{}

\title{
  Week 8 Discussion
}
\author{CAS CS 132: Geometric Algorithms}
\date{October 23, 2023}

\begin{document}
\maketitle

\noindent During discussion sections, we will go over three problems.
\begin{itemize}
\item The first will be a warm-up question, to help you verify your understanding of the material.
\item The second will be a solution to a problem on the assignment of the previous week.
\item The third will be a problem similar to one on the assignment of the following week.
\end{itemize}
The remainder of the time will be dedicated to open Q\&A.

\pagebreak
\section{Warm up}
Consider the following matrices in echelon form and reduced echelon form.
\begin{displaymath}
  A =
  \begin{bmatrix}
    2 & -3 & 0 & 0\\
    0 & 0 & 1 & 2 \\
    0 & 0 & 0 & 10
  \end{bmatrix}
  \quad
  B =
  \begin{bmatrix}
    1 & 0 & 2 & 0 & 3 \\
    0 & 1 & 3 & 0 & -1 \\
    0 & 0 & 0 & 1 & 0 \\
    0 & 0 & 0 & 0 & 0
  \end{bmatrix}
  \quad
  C =
  \begin{bmatrix}
    10 & 1 & 3 \\
    0 & 1 & -3 \\
    0 & 0 & 6
  \end{bmatrix}
\end{displaymath}

\begin{enumerate}
\item If $D \sim A$ (that is $D$ is row equivalent to $A$) and $D$ is the augmented matrix of a system of linear equations, what can we say about the solutions of the system? (e.g., general form, number of solutions)
\item If $D \sim C$, then what can we say about the columns of $A$? (e.g., linearly independence, span)
\item If $D \sim C$, then what can we say about the solutions to the equation $D\mathbf x = \mathbf 0$?
\item If $D \sim B$, then what can we say about the columns of $D$?
\item If $D \sim B$ and $D$ represents the augmented matrix of a system of linear equations of $B$, what can we say about the columns of the coefficient matrix for the system?
\item If $D \sim A$ and $D$ is the augmented matrix of a system of linear equations, what can we say about the solutions of the system for which $x_1 = 0$.

\end{enumerate}

\medskip

\begin{solution}
\end{solution}

\pagebreak
\section{Composing Rotations}
Consider the following $\mathbb R^3$ rotation matrices.
\begin{displaymath}
  A =
  \begin{bmatrix}
    \cos 45^\circ & -\sin 45^\circ & 0 \\
    \sin 45^\circ & \cos 45^\circ & 0 \\
    0 & 0 & 1
  \end{bmatrix}
  \qquad
  B =
  \begin{bmatrix}
    1 & 0 & 0 \\
    0 & \cos 180^\circ & -\sin 180^\circ \\
    0 & \sin 180^\circ & \cos 180^\circ
  \end{bmatrix}
\end{displaymath}
The matrix $A$ rotates vectors around the $x_3$-axis by 45 degrees, and $B$ rotates vectors around the $x_1$-axis by 180 degrees.
Also remember that
\begin{align*}
  \cos 45^\circ &= \frac{\sqrt{2}}{2} \\
  \sin 45^\circ &= \frac{\sqrt{2}}{2} \\
  \cos 180^\circ &= -1 \\
  \sin 180^\circ &= 0 \\
  \cos(-\theta) &= \cos(\theta) \text{ for any $\theta$} \\
  \sin(-\theta) &= -\sin(\theta) \text{ for any $\theta$} \\
\end{align*}
\begin{enumerate}
\item (4 points) Determine $A^{-1}$.
  Every entry should be a scalar multiple of $1$, $\cos 45^\circ$ or $\sin 45^{\circ}$.
  In particular, don't include any trigonometric functions applied to negative angles.
  \textit{Hint.} You don't have to do any calculations for this.
  Think about what it means to ``undo'' a rotation.
\item (6 points) Calculate $ABA^{-1}$.
  Your solutions should be as reduced as possible, and should not contain any trigonometric functions.
\item (4 points) Describe what the transformation implemented by $ABA^{-1}$ does geometrically. Your answer should be of the form ``rotation by [NUMBER] degrees around the span of [VECTOR].'' \textit{Hint.} We've seen this transformation before.
\end{enumerate}

\medskip

\begin{solution}
\end{solution}

\pagebreak
\section{Regular Stochastic Matrices}

For each of the following stochastic matrices:
\begin{itemize}
\item Determine if it is regular.
\item Write down a solution to the homogeneous equation $(A - I)\mathbf x = \mathbf 0$.
\item If it is possible, find a steady-state vector from this solution.
\end{itemize}

\begin{enumerate}
\item
  \begin{displaymath}
    \begin{bmatrix}
      1 & 0.5 \\
      0 & 0.5
    \end{bmatrix}
  \end{displaymath}
\item
  \begin{displaymath}
    \begin{bmatrix}
      0.2 & 1 \\
      0.8 & 0
    \end{bmatrix}
  \end{displaymath}
\end{enumerate}

\medskip

\begin{solution}
\end{solution}
\end{document}
