\documentclass{article}

% packages for math
\usepackage{amsthm}
\usepackage{amsmath}
\usepackage{amssymb}
\usepackage{amsfonts}

% package for including images
\usepackage{graphicx}

% TAKEN FROM OVERLEAF DOCUMENTATION
% https://www.overleaf.com/learn/latex/Code_listing
\usepackage{listings}
\lstset{language=Python}
\usepackage{xcolor}
\definecolor{codegreen}{rgb}{0,0.6,0}
\definecolor{codegray}{rgb}{0.5,0.5,0.5}
\definecolor{codepurple}{rgb}{0.58,0,0.82}
\definecolor{backcolour}{rgb}{0.95,0.95,0.92}
\lstdefinestyle{mystyle}{
    backgroundcolor=\color{backcolour},
    commentstyle=\color{codegreen},
    keywordstyle=\color{magenta},
    numberstyle=\tiny\color{codegray},
    stringstyle=\color{codepurple},
    basicstyle=\ttfamily\footnotesize,
    breakatwhitespace=false,
    breaklines=true,
    captionpos=b,
    keepspaces=true,
    numbers=left,
    numbersep=5pt,
    showspaces=false,
    showstringspaces=false,
    showtabs=false,
    tabsize=2
}
\lstset{style=mystyle}

% environment for solutions
\theoremstyle{remark}
\newtheorem*{solution}{Solution}

% capital letters for problem parts
\renewcommand{\theenumi}{\Alph{enumi}}

% no page numbers
\pagenumbering{gobble}

% UNCOMMENT IF YOU DON'T WANT PROBLEMS ON INDIVIDUAL PAGES
% \renewcommand{\pagebreak}{}

\title{
  Midterm Review
}
\author{CAS CS 132: Geometric Algorithms}
\date{}

\begin{document}
\maketitle

\pagebreak
\section{Solving Systems of Linear Equations}
Find a solution the following system of linear equations.
% 0
\begin{align*}
  x_1 + x_2 - x_3 &= -9 \\
  x_2 - 2x_3 &= -1 \\
  x_1 + x_2 &= -10 \\
\end{align*}

\begin{solution}
\end{solution}

\pagebreak
\section{LAA 1.2.18}
Determine all values of $h$ such that the following matrix is the augmented matrix of a consistent linear system.
\begin{displaymath}
  \begin{bmatrix}
    1 & -3 & -2 \\
    5 & h & -7
  \end{bmatrix}
\end{displaymath}

\medskip

\begin{solution}
\end{solution}

\pagebreak
\section{General Form Solutions}
Consider the following system of linear equations.
% 10
\begin{align*}
  x_1 - 5x_2 - x_3 - 2x_4 &= 3 \\
  x_3 - 2x_4 &= 11 \\
  (-2)x_3 + 5x_4 &= -24 \\
\end{align*}
\begin{enumerate}
\item Write down a general form solution which describes the solution set of the following system of linear equations.
\item Write down a different general form solution which describes the same solution set (i.e., one in which a different variable is free).
\end{enumerate}

\medskip

\begin{solution}
\end{solution}

\pagebreak
\section{LAA 1.3.13}
Consider the following matrix $A$ and vector $\mathbf b$.
\begin{displaymath}
  A =
  \begin{bmatrix}
    1 & -4 & 2 \\
    0 & 3 & 5 \\
    -2 & 8 & -4
  \end{bmatrix}
  \qquad
  \mathbf b =
  \begin{bmatrix}
    3 \\ -7 \\ h
  \end{bmatrix}
\end{displaymath}
\begin{enumerate}
\item Determine if $\mathbf b$ can be written as a linear combination of the columns of $A$ if $h = -3$.
\item For what values of $h$ can $\mathbf b$ be written as a linear combination of the columns of $A$.
\end{enumerate}

\medskip

\begin{solution}
\end{solution}

\pagebreak
\section{Intersections of Spans}
Consider the following vectors.
\begin{displaymath}
  \mathbf v_1 =
  \begin{bmatrix}
    1 \\ 0 \\ 0
  \end{bmatrix}
  \quad
  \mathbf v_2 =
  \begin{bmatrix}
    0 \\ 1 \\ 0
  \end{bmatrix}
  \quad
  \mathbf v_3 =
  \begin{bmatrix}
    1 \\ 2 \\ -1
  \end{bmatrix}
  \quad
  \mathbf v_4 =
  \begin{bmatrix}
    0  \\ 3 \\ 2
  \end{bmatrix}
\end{displaymath}
Find a nonzero vector which lies in both $\mathsf{span}\{\mathbf v_1, \mathbf v_2\}$ and $\mathsf{span}\{\mathbf v_3, \mathbf v_4\}$.

\medskip

\begin{solution}
\end{solution}

\pagebreak
\section{LAA 1.7.31}
Find a nontrivial solution to matrix equation $A \mathbf x = \mathbf 0$ without performing any row reductions. (Hint. What is the relationship between the first two columns and the last column of $A$?)
\begin{displaymath}
  A =
  \begin{bmatrix}
    2 & 3 & 5 \\
    -5 & 1 & -4 \\
    -3 & -1 & -4 \\
    1 & 0 & 1
  \end{bmatrix}
\end{displaymath}

\medskip

\begin{solution}
\end{solution}

\pagebreak
\section{Linearly Independent Vectors}
Consider three arbitrary vectors
\begin{displaymath}
  \mathbf v =
  \begin{bmatrix}
    v_1 \\ v_2 \\ v_3
  \end{bmatrix}
  \quad
  \mathbf w =
  \begin{bmatrix}
    w_1 \\ w_2 \\ w_3
  \end{bmatrix}
  \quad
  \mathbf u =
  \begin{bmatrix}
    u_1 \\ u_2 \\ u_3
  \end{bmatrix}
\end{displaymath}
and suppose that they are linearly independent.
\begin{enumerate}
\item What is the maximum number of entries of these vectors which can be $0$? Note that the solution will be a number between $0$ and $9$.
\item What is the minimum number?
\end{enumerate}
In each case provide an example.

\medskip

\begin{solution}
\end{solution}

\pagebreak
\section{Drawing Linear Transformations}
Draw the unit square after being transformed by the matrix transformation implemented by
\begin{displaymath}
  \begin{bmatrix}
    1 & 2 \\
    3 & 4
  \end{bmatrix}
\end{displaymath}

\medskip

\begin{solution}
\end{solution}

\pagebreak
\section{Matrices of Linear Transformations}
Find the matrix which implemented the following transformation.
\begin{displaymath}
  \begin{bmatrix}
    v_1 \\ v_2 \\ v_3 \\ v_4
  \end{bmatrix}
  \mapsto
  \begin{bmatrix}
    v_2 \\ v_1 \\ v_3 + v_4
  \end{bmatrix}
\end{displaymath}
\medskip

\begin{solution}
\end{solution}

\pagebreak
\section{3D Linear Transformations}
Considering the transformation $T$ implemented by the following matrix.
\begin{displaymath}
  \begin{bmatrix}
    \cos2 & 0 & -\sin2 \\
    0 & 1 & 0 \\
    \sin2 & 0 & \cos2
  \end{bmatrix}
\end{displaymath}
Describe geometrically what $T$ does.
Then find a vector $\mathbf v$ whose span is not changed by this transformation (i.e., $\mathsf{span}\{\mathbf v\} = \mathbf{span}\{T(\mathbf v)\}$).

\medskip

\begin{solution}
\end{solution}

\end{document}
