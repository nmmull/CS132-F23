\documentclass{article}

% packages for math
\usepackage{amsthm}
\usepackage{amsmath}
\usepackage{amssymb}
\usepackage{amsfonts}

% package for including images
\usepackage{graphicx}

% TAKEN FROM OVERLEAF DOCUMENTATION
% https://www.overleaf.com/learn/latex/Code_listing
\usepackage{listings}
\lstset{language=Python}
\usepackage{xcolor}
\definecolor{codegreen}{rgb}{0,0.6,0}
\definecolor{codegray}{rgb}{0.5,0.5,0.5}
\definecolor{codepurple}{rgb}{0.58,0,0.82}
\definecolor{backcolour}{rgb}{0.95,0.95,0.92}
\lstdefinestyle{mystyle}{
  backgroundcolor=\color{backcolour},
  commentstyle=\color{codegreen},
  keywordstyle=\color{magenta},
  numberstyle=\tiny\color{codegray},
  stringstyle=\color{codepurple},
  basicstyle=\ttfamily\footnotesize,
  breakatwhitespace=false,
  breaklines=true,
  captionpos=b,
  keepspaces=true,
  numbers=left,
  numbersep=5pt,
  showspaces=false,
  showstringspaces=false,
  showtabs=false,
  tabsize=2
}
\lstset{style=mystyle}

% environment for solutions
\theoremstyle{remark}
\newtheorem*{solution}{Solution}

% capital letters for problem parts
\renewcommand{\theenumi}{\Alph{enumi}}

% no page numbers
\pagenumbering{gobble}

% UNCOMMENT IF YOU DON'T WANT PROBLEMS ON INDIVIDUAL PAGES
% \renewcommand{\pagebreak}{}

\newcommand{\vv}[1]{\mathbf{#1}}
\newcommand{\R}{\mathbb R}
\DeclareMathOperator{\vspan}{span}
\DeclareMathOperator{\cod}{cod}
\DeclareMathOperator{\ran}{ran}
\DeclareMathOperator{\col}{Col}
\DeclareMathOperator{\nul}{Nul}
\DeclareMathOperator{\rank}{rank}

\title{
  Homework 11
}
\author{CAS CS 132: Geometric Algorithms}
\date{Due: \textbf{Thursday December 7, 2023 at 11:59PM}}

\begin{document}
\maketitle

\subsection*{Submission Instructions}
\begin{itemize}
\item Make the answer in your solution to each problem abundantly clear (e.g., put a box around your answer or used a colored font if there is a lot of text which is not part of the answer).
\item Choose the correct pages corresponding to each problem in Gradescope. Note that Gradescope registers your submission as soon as you submit it, so you don't need to rush to choose corresponding pages.
  \textbf{For multipart questions, please make sure each part is accounted for.}
\end{itemize}
Graders have license to dock points if either of the above instructions are not properly followed.


\section*{Practice Problems}

The following list of problems comes from \textit{Linear Algebra and its Application 5th Ed} by David C.\ Lay, Steven R.\ Lay, and Judi J.\ McDonald.
They may be useful for solidifying your understanding of the material and for studying in general.
\textbf{They are optional, so please don't submit anything for them}.

\begin{itemize}
\item 6.1.5-8, 6.1.17-18, 6.1.19-20
\item 6.2.4-6, 6.2.12-13, 6.2.23-24
\item 6.3.3-5, 6.3.11-12, 6.3.21-22
\item 6.5.2-4, 6.5.5, 6.5.17-18
\end{itemize}

\pagebreak
\section{Basic Analytic Geometry}
\begin{displaymath}
  \vv v =
  \begin{bmatrix}
    1 \\ -1 \\ 5 \\ 7 \\ 4
  \end{bmatrix}
  \qquad
  \vv u =
  \begin{bmatrix}
    3 \\ 1 \\ 3 \\ -1 \\ 0
  \end{bmatrix}
\end{displaymath}
You must show your work for all of the following calculations.
\begin{enumerate}
\item (3 points) Compute the norm of $\vv v$.
\item (4 points) Find the unit vector in the direction of $\vv u$.
\item (4 points) Compute the distance between $\vv u$ and $\vv v$.
\item (4 points) Compute approximately the angle between $\vv u$ and $\vv v$ (you will need to use a calculator or Python).
\item (5 points) Using the values you have computed so far, and without computing the angle directly, determine approximately the angle between $\vv v$ and $\vv v - \vv u$. Justify your answer. (\textit{Hint.} Think about triangle created by connecting the tips of $\vv u$, $\vv v$ by a line segment.)
\end{enumerate}

\medskip

\begin{solution}
\end{solution}

\pagebreak
\section{The Gram-Schmidt Process}

\begin{displaymath}
  \vv v_1 =
  \begin{bmatrix}
    1 \\ 0 \\ 1 \\ 1
  \end{bmatrix}
  \qquad
  \vv v_2 =
  \begin{bmatrix}
    0 \\ 1 \\ 1 \\ 2
  \end{bmatrix}
  \qquad
  \vv v_3 =
  \begin{bmatrix}
    1 \\ 0 \\ 2 \\ -2
  \end{bmatrix}
  \qquad
  \vv u =
  \begin{bmatrix}
    2 \\ -2 \\ 1 \\ -5
  \end{bmatrix}
\end{displaymath}
You must show your work for all of the following calculations.\footnote{
The Gram-Schmidt process is an algorithm for converting a basis into an orthogonal basis.
We won't have time to cover Gram-Schmidt in full detail in this course, but this problem details how the procedure works for three vectors.}
\begin{enumerate}
\item (4 points) Find the component of $\vv v_2$ orthogonal to $\vv v_1$ (that is, find the vector $\vv z$ such that $\vv v_2 = \hat{\vv v_2} + \vv z$ where $\hat{\vv v_2}$ is the orthogonal projection of $\vv v_2$ onto $\vv v_1$). We will refer to this as $\vv v_2'$ below.
\item (4 points) Find the component of $\vv v_3$ orthogonal to $\vv v_1$. We will refer to this as $\vv v_3'$ below.
\item (4 points) Find the component of $\vv v_3'$ orthogonal to $\vv v_2'$. We will refer to this as $\vv v_3''$ below.
\item (3 points) Compute the inner product $\vv v_3''$ and $\vv v_1$.
\item (5 points) Find $[\vv u]_{\mathcal B}$ where $\mathcal B = \{\vv v_1, \vv v_2', \vv v_3''\}$
\end{enumerate}
\medskip

\begin{solution}
\end{solution}
\vfill

\pagebreak
\section{Orthogonal Projection Matrices}

\begin{displaymath}
  \vv v =
  \begin{bmatrix}
    2 \\ 3 \\ 1
  \end{bmatrix}
  \qquad
  A =
  \begin{bmatrix}
    1 & 2 & -5 \\
    1 & 1 & 3 \\
    8 & 14 & -6 \\
    1 & 2 & 1
  \end{bmatrix}
  \qquad
  B =
  \begin{bmatrix}
    1 & -4 & -2 & 9 & 0 \\
    0 & 2 & 2 & -6 & 1 \\
    3 & -5 & 1 & 9 & 5 \\
    0 & 1 & 1 & -2 & 1
  \end{bmatrix}
\end{displaymath}

You must show your work for all of the following calculations.
\begin{enumerate}
\item (2 points) Find the matrix which implements orthogonal projection onto $\vspan\{\vv v\}$.
\item (3 points) For an arbitrary vector $\vv x$ in $\R^n$, find an expression for the matrix which implements orthogonal projection onto $\vspan\{\vv x\}$.
\item (3 points) Find approximately the matrix which implements orthogonal projection onto $\col A$.
  You should use Python for this.
  Write down the NumPy expressions you used for your calculation.
\item (6 points) Find approximately the matrix which implements orthogonal projection onto $\col B$.
  You should use Python for this. (\textit{Hint.} First determine the rank of $B$.)
  Write down the NumPy expressions you used for your calculation.
\item (3 points) What is the relationship between $\col A$ and $\col B$? Justify your answer.
\item (3 points) Suppose that $C$ is an arbitrary $m \times n$ \textit{orthonormal} matrix. Find an expression for the matrix which implements orthogonal projection onto $\col C$. Your expression should be as simplified as possible.
\end{enumerate}

\medskip

\begin{solution}
\end{solution}

\pagebreak
\section{Least Squares}

\begin{displaymath}
  A =
  \begin{bmatrix}
    1 & 2 \\
    0 & -1 \\
    1 & -1
  \end{bmatrix}
  \qquad
  \vv b =
  \begin{bmatrix}
    4 \\ 2 \\ 0
  \end{bmatrix}
  \qquad
  U =
  \begin{bmatrix}
    1 & 0 & 1 \\
    1 & 0 & 1 \\
    1 & 0 & 1 \\
    1 & 1 & 0 \\
    1 & 1 & 0 \\
    1 & 1 & 0 \\
  \end{bmatrix}
  \qquad
  \vv c =
  \begin{bmatrix}
    6 \\ 5 \\ 4 \\ 7 \\ 2 \\ 3
  \end{bmatrix}
\end{displaymath}

\begin{enumerate}
\item (3 points) Find the normal equations for $A\vv x = \vv b$.
\item (2 points) Find a least squares solution for $A \vv x = \vv b$. Then use \texttt{np.linalg.lstsq} to verify your solution. You should read the NumPy documentation to verify how this function works. Include in your solution what Python code your wrote to verify your solution.
\item (1 points) Is the least squares solution you found in the previous part unique? Justify your answer.
\item (4 points) Find the normal equations for $U\vv x = \vv c$.
\item (4 points) Find two linearly independent least squares solutions for $U\vv x = \vv c$.
\item (6 points) Let $C$ be an arbitrary $m \times n$ matrix of rank $k$ and let $\mathbf b$ be an arbitrary vector in $\R^m$.
  Find an expression for the maximum size of a linearly independent set of least squares solutions for $C\mathbf x = \mathbf b$ where $\vv b \not = \vv 0$.
  You do not need to prove that your expression is correct, but you should justify your answer.
  (\textit{Hint.} Think about how to generate multiple least squares solutions in the case that there is more than one. Your expression should be in terms of $m$, $n$, and $k$, thought it may not use every one of these values.)
\end{enumerate}
\medskip

\begin{solution}
\end{solution}
\end{document}
